\documentclass{beamer}
\usefonttheme{professionalfonts}
\usepackage[UTF8,space,hyperref]{ctex}
\usepackage{graphics}
\usepackage{wrapfig}
\usepackage{ulem}
\usetheme{Berlin}
\usepackage{tikz}
\usepackage{caption}
\usepackage{ulem}
\usepackage{subfigure}
\usecolortheme{default}
\setbeamertemplate{headline}{} 

\title{Data structures Final Assignment}
\author{Yaoming Shi, Hanlin Qian, Yizhou Xue}
\institute{UCAS}
\date{2025.6.25}

\begin{document}

\maketitle

\begin{frame}{题目要求}
    \begin{figure}
        \centering
        \includegraphics[width=0.9\textwidth]{figures/require.png}
    \end{figure}
\end{frame}

\begin{frame}{总体设计与功能}
  \begin{itemize}
    \item 支持火车/飞机数据的批量导入。
    \item 火车建模:分层图最短路,支持不同优先级。
    \item 飞机建模:直接对起点和终点之间连边,跑最短路。
  \end{itemize}
  \begin{figure}
    \centering
    \includegraphics[width=0.5\textwidth]{figures/train_data.png}
  \end{figure}
\end{frame}

\begin{frame}{火车部分建模与查询}
  \begin{itemize}
    \item \textbf{分层图建模}:每个车站每个车次拆点,分两层,支持换乘建边。
    \item \textbf{边权设计}:一类边为区间时间/票价,二类边为换乘间隔。
    \item \textbf{最短路算法}:Dijkstra,支持时间优先/价格优先。
    \item \textbf{换乘规则}:可设定是否允许换乘及最小换乘间隔。
    \item \textbf{查询接口}:支持城市/车站为起终点,输出详细方案。
  \end{itemize}
\end{frame}

\begin{frame}{飞机部分建模与查询}
  \begin{itemize}
    \item \textbf{点对点建图}:每个城市为节点,航班为有向边。
    \item \textbf{支持一次中转}:简单bfs,允许中转查询。
    \item \textbf{多关键字查询}:可选时间优先或价格优先。
    \item \textbf{批量导入}:自动识别城市名并建表。
    \item \textbf{详细输出}:每段航班号、起降城市、时间、价格。
  \end{itemize}
\end{frame}

\begin{frame}{数据结构与核心实现}
  \begin{itemize}
    \item \textbf{哈希表}:城市名、站名映射唯一 ID,支持中文。
    \item \textbf{链式前向星}:存储火车分层图、飞机航班图。
    \item \textbf{小根堆}:Dijkstra 最短路,支持多关键字比较。
    \item \textbf{全局变量}:统一管理所有城市、站点、图结构。
    \item \textbf{失效标记}:通过expire字段实现逻辑删除/恢复。对每个 Station 和 Train\_id 维护一个 expire 字段,表示该车次是否失效。
  \end{itemize}
\end{frame}

\begin{frame}{用户交互界面}
    \begin{figure}
        \centering
        \includegraphics[width=0.5\textwidth]{figures/app.png}
        \caption*{控制台交互界面}
    \end{figure}
\end{frame}

\begin{frame}{运行结果———火车车次加入与查询}
    \begin{figure}
        \centering
        \includegraphics[width=0.8\textwidth]{figures/result1.png}
    \end{figure}
\end{frame}

\begin{frame}{运行结果——航班加入与查询}
    \begin{figure}
        \centering
        \includegraphics[width=0.6\textwidth]{figures/result2.png}
    \end{figure}
\end{frame}

\begin{frame}{主要代码结构与模块说明}
  \begin{itemize}
    \item \textbf{main.c}:主入口,调用app初始化与主循环。
    \item \textbf{app.c/h}:系统初始化与交互主循环。
    \item \textbf{global.c/h}:全局变量与初始化。
    \item \textbf{modify.c/h}:增删查改接口,自动建站/建城。
    \item \textbf{train.c}:火车数据导入、最短路与方案输出。
    \item \textbf{plane.c/h}:航班数据导入、查询与增删。
    \item \textbf{graph.c/h}:分层图、节点、边、Dijkstra实现。
    \item \textbf{heap.c/h}:小根堆,支持多关键字优先级。
  \end{itemize}
\end{frame}

\begin{frame}{总结与体会}
  \begin{itemize}
    \item 本系统实现了复杂的交通网络建模与多条件最优路径查询。
    \item 采用分层图、哈希表、堆等多种数据结构,兼顾效率与灵活性。
    \item 支持中文输入输出,批量导入,用户体验友好。
    \item 工程结构复杂,模块划分清晰,便于维护与扩展。
    \item 收获了大规模数据结构设计与工程实现的宝贵经验。
  \end{itemize}
\end{frame}

\end{document}
